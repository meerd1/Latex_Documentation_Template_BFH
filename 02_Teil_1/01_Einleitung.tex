\chapter{Einleitung}
\label{chap:teil1_einleitung}

Dieses Dokument dient einerseites zur Illustration der \LaTeX{} Vorlage anhand des Corporate Designs\index{Corporate Design} der Berner Fachhochschule\index{Berner Fachhochschule} und andererseits als Anleitung f�r deren Verwendung. Dabei wird vorausgesetzt, dass der Benutzer bereits Erfahrungen mit \LaTeX{} besitzt oder gewillt ist, sich w�hrend der Benutzung in das Thema einzuarbeiten. Im Quellenverzeichnis sind einige n�tzliche Eintr�ge zu diversen B�chern und Dokumenten im Internet �ber \LaTeX{} zu finden.

% Eintr�ge im Verzeichnis erscheinen lassen ohne hier eine Referenz einzuf�gen
\nocite{kopka:band1}
\nocite{raichle:bibtex_programmierung}
\nocite{MiKTeX}
\nocite{KOMA}
\nocite{TeXnicCenter}
\nocite{Marti06}
\nocite{Erbsland08}
\nocite{juergens:einfuehrung}
\nocite{juergens:fortgeschritten}


\section{Dokumentaufbau}
\label{sec:teil1_einleitung_aufbau}

Das vorliegende Dokument ist so aufgebaut wie die Dokumentation einer Projektarbeit oder Thesis\index{Thesis}. Im Kapitel \ref{chap:teil2_anleitungen} werden die verwendeten Pakete kurz erkl�rt und Hinweise gegeben, wie die Bibliographie und das Glossar zu verwenden sind. Kapitel \ref{chap:teil3_satzspiegeltest} stellt ein reines Beispielkapitel dar, um den Satzspiegel zu pr�fen.


\section{Kontakt}
\label{sec:teil1_einleitung_kontakt}

Die Hersteller dieser Vorlage sind nat�rlich froh um Verbesserungsvorschl�ge jeder Art.

\begin{table}[H]
	\centering
	\begin{tabular}{lll} \toprule
		\textbf{Vorname Name} & \textbf{E-Mail} & \textbf{Funktion} \\ \midrule
		Roger Weber & roger.weber@bfh.ch & Auftraggeber \\ \midrule
		David Burri & david.burri@bfh.ch & Erstellung der Bachelor-Thesis Vorlage \\ \midrule
		Tobias R�etschi & tobias.rueetschi@bfh.ch & Erstellung der Skript-Vorlage \\ \bottomrule
	\end{tabular}
	\caption{Kontaktpersonen}
	\label{tab:teil1_kontaktpersonen}
\end{table}
