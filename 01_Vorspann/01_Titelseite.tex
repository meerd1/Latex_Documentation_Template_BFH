
% Project documentation template
% ===========================================================================
% This is part of the document "Project documentation template".
% Authors: brd3
%

\begin{titlepage}


% Red Bar and BFH-Logo absolute placed at (87,10) on A4
% Actually not a realy satisfactory solution but working.
%---------------------------------------------------------------------------
\setlength{\unitlength}{1mm}
\begin{textblock}{210}(-10,-10)
	\begin{picture}(210,32)
		\put(0,0){\color{bfhred}\rule{240mm}{30mm}}
	\end{picture}
\end{textblock}

\begin{textblock}{0}[0,0](84,7)
	\includegraphics[scale=1.0]{Bilder/bfh_de_white.pdf}
\end{textblock}

% Titel / Untertitel / Autor:
%---------------------------------------------------------------------------
\begin{flushleft}

\vspace*{2cm}

\fontsize{18pt}{20pt}\selectfont
Modul BTExxxx \\
\fontsize{12pt}{15pt}\selectfont\vspace{0.5em}
Fachbereich Elektro- und Kommunikationstechnik

\vspace{2cm}

\fontsize{30pt}{32pt}\selectfont 
\noindent \textcolor{titlecolor}{\textbf{\title}} \\

\vspace{10cm}
\fontsize{12pt}{15pt}\selectfont
\begin{tabbing}
xxxxxxxxxxxxxxx\=xxxxxxxxxxxxxxxxxxxxxxxx	\kill
Author:				\> \author									\\
Datum:				\> \today										\\
\end{tabbing}
\end{flushleft}

\vspace{1cm}
\fontsize{12pt}{15pt}\selectfont
Damit an der Berner Fachhochschule im Bereich der technischen Informatik Module abgehalten werden k�nnen, braucht es im Normalfall ein Skript. Da diese Skripte immer wieder �berarbeitet werden sollten, wird mit diesem Dokument eine Vorlage geliefert, welche das ganze auf \LaTeX{} aufbauend einheitlich macht.

\end{titlepage}

%
% ===========================================================================
% EOF
%
